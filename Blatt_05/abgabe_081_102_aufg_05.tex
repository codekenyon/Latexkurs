% Rieke, Ken    4234782
% Thill, Gregor 4260617

\documentclass[12pt,a4paper]{scrartcl}
\usepackage[utf8]{inputenc}
\usepackage[T1]{fontenc}
\usepackage[ngerman]{babel}
\usepackage{amsmath,amssymb}

%%%%%%%%%%%%%%%%%%
% Eigene Befehle %
%%%%%%%%%%%%%%%%%%

% Zeichenfolge det soll als mathematischer Operator gelten:
\DeclareMathOperator{\dete}{det}

% \Vektor erzeugt eine Vektordarstellung für die Variable #2 % von Element 1
% bis zum optionalen Index #1 (n, wenn nicht angegeben).
\newcommand{\Vektor}[2][n]{\vec{#2} = (#2_1, \ldots, #2_{#1})}

% \DProdukt erzeugt ein doppeltes Produktformelzeichen aus den Zählvariablen
% #1 und #2 bis zur Indexvariablen #3.
\newcommand{\DProdukt}[3]{\prod\limits_{#1=1}^{#3} \prod\limits_{#2=1}^{#1-1}}

% \SProdukt erzeugt ein Produktformelzeichen mit mehrzeiligem Index, welcher
% aus den Zählvariablen #1 und #2 plus eine Indexvariable #3 besteht.
\newcommand{\SProdukt}[3]{\prod_{\substack{#1,#2\in[#3]\\#1>#2}}}

% Die Beschreibung zu \Vand ergibt sich aus dem Aufgabentext von Übungsblatt 5.
% Der optionale Parameter #1 gibt den Index an, während der notwendige Parameter
% #2 die Variable der Einträge angibt.
\newcommand{\Vand}[2][n]{
  \left(
  \begin{array}{ccccc}
    1      & #2_1    & #2_1^2    & \cdots & #2_1^{#1-1}    \\
    1      & #2_2    & #2_2^2    & \cdots & #2_2^{#1-1}    \\
    \vdots & \vdots  & \vdots    & \ddots & \vdots         \\
    1      & #2_{#1} & #2_{#1}^2 & \ldots & #2_{#1}^{#1-1}
  \end{array}
  \right)
}

%%%%%%%%%%%%%%%%%%%%%%%%
% Ende eigener Befehle %
%%%%%%%%%%%%%%%%%%%%%%%%

\begin{document}

\section{Eine besondere Determinante}
  Für $\Vektor[k]{x}$ ist
  \begin{equation}
    V(\vec{x}) := \Vand[k]{x} 
  \end{equation}
  die Vandermonde-Matrix zu $\vec{x}$. Es sei $[n] := \{1,\ldots,n\}.$ Mittels
  \begin{align}
    \dete V(\vec{x}) & = |V(\vec{x})| \nonumber        \\
    \label{ProdProd} & = \DProdukt{i}{j}{n}(x_i - x_j) \\
    \label{Prod}     & = \SProdukt{i}{j}{n}(x_i - x_j)
  \end{align}
  lässt sich die Determinante einer Vandermonde-Matrix bestimmen. Die Formeln
  mit den Nummern \ref{ProdProd} und \ref{Prod} stellen dabei nur unterschiedliche
  Schreibweisen dar.

  Es sei nun $\Vektor{\alpha}$ mit paarweise verschiedenen reellen Zahlen
  $\alpha_i$, d.h. $\alpha_i \neq \alpha_j$ für alle $i \neq j$. Demnach muss
  für $i, j$ mit $i \neq j$ der Zusammenhang
  \begin{align}
    \alpha_{i}                      & \neq \alpha_j \qquad | - \alpha_j \nonumber \\
    \Rightarrow \alpha_i - \alpha_j & \neq 0 \nonumber
  \end{align}
  gelten. Für den Wert der Determinante der zugehörigen Vandermonde-Matrix, d.h. für
  \begin{align}
    \underline{\dete V(\vec\alpha)} = \dete \Vand{\alpha}
      & = \left| \Vand{\alpha}\right| \nonumber 
    \intertext{folgt entsprechend}  \nonumber \\
    \label{AlphaProd}
      & \stackrel{(\ref{Prod})}{=}
        \SProdukt{i}{j}{n}
        \overbrace{(\alpha_i - \alpha_j)}^{\neq 0}
        \underline{\neq 0}
  \end{align}
  und damit die Existenz ihrer Inversen $V(\vec{\alpha})^{-1}$. Die etwas strengere Bedinung
  \[
    |V(\vec{\alpha})|\not\thickapprox 0
  \]
  ist für eine numerisch stabile Bestimmung der Inversen vorzuziehen.\newpage

  Dieses Ergebnis ist recht spannend. So wissen wir nun aus Formel (\ref{AlphaProd})
  auf Seite \pageref{AlphaProd}, dass für 42 beliebige paarweise verschiedene
  Zahlen $a_i$ die an dieser Stelle linksbündig gesetzte Matrix
  \begin{flalign*}
    \Vand[42]{a} &&
  \end{flalign*}
  regulär ist.

\section{Ein spezieller Fall}
  Es soll nun der Fall $k = 2$, d.h. die Vandermonde-Matrix $V(\vec{x}) = V(x_1,x_2)$
  betrachtet werden. Entsprechend muss dann für deren Determinante
  \begin{align}
    \dete V(x_1,x_2) =
    \begin{cases}
      0 & \text{für }x_1=x_2\\
      x_2 - x_1 \neq 0 & \text{für } x_1 \neq x_2 \nonumber
    \end{cases}
  \end{align}
  gelten.

\end{document}
