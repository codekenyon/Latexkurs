% Rieke, Ken    4234782
% Thill, Gregor 4260617

% Bitte fügen Sie am Ende des Dokuments an den entsprechenden Stellen Ihre Tabelle und Ihren Text ein.
% Denken Sie an Ihre Namen und Martikelnummer sowie an die korrekte DATEIBENENNUNG!


\documentclass[a4paper,12pt]{scrartcl}

\usepackage[utf8]{inputenc}
\usepackage[T1]{fontenc}
\usepackage[ngerman]{babel}
% Fehler: Schreibfehler. documert statt document
\begin{document}

\section{Fehlerteil}

% Fehler: Fehlende Klammerung am Ende von \textit. (})
\textit{Ich hoffe, die heutige Vorlesung war spannend und Sie haben viel gelernt.
Wir haben uns ausführlich mit \textbf{Boxen}, mit \textbf{Abständen},
mit \textbf{Tabulatoren} und mit \textbf{Tabellen beschäftigt.
Besonders spannend sollte auch das Thema Fehlerbehebung gewesen sein,
da dieses Thema besonders wichtig und hilfreich ist.}}

\begin{tabbing}
Tabulatoren sind \= schön. \\
\> gut. \\
\> vielseitig.\\
%Fehler: Fehlender Zeilenumbruch. \\
\> manchmal nervig. \\
\> nicht so toll wie Tabellen.
\end{tabbing}

\noindent Abschließend listen wir nun nochmal die Superhelden auf,
die wir in der aktuellen Vorlesung schon kennengelernt haben:
\begin{itemize}
\item Iron Man
\item Captain America
%Fehler: Schreibfehler. /item statt /ietm
\item Thor
\item Hulk
\end{itemize}


\newpage
\section{Meine Tabelle}

% Hier fügen Sie Ihre Tabelle ein.
% [X] Tabelle passt auf hochformatige Seite und ragt nicht über
% [X] Die Tabelle ist mit vertikalen und horizontalen Linien berandet
% [X] Mehrere Spalten werden in einzelne Zeilen zusammengefasst
% [X] Die Tabelle soll zentriert sein
% [X] Es sind mindestens eine rechtsbündige, eine linksbündige und eine Spalte mit normaler Fließtextformatierung enthalten

\begin{center}
  \begin{tabular}{|l|r|p{8cm}|}
  \hline
  \multicolumn{3}{|c|}{Stadien körperlicher Auskühlung und ihre Symptome}  \\
  \hline \hline
   & Temp. unter & Symptome  \\ \hline
             &              & Kältegefühl, Frieren \\ \cline{3-3}
             & 37 $^\circ$C & Finger und Zehen werden zunehmend gefühllos (verminderte Blutzufuhr) \\ \cline{2-3}
             &              & Bewegungen werden ungelenk \\ \cline{3-3}
  1. Stadium & 36 $^\circ$C & starkes, unkontrollierbares Zittern \\ \cline{3-3}
             &              & Müdigkeit \\ \cline{2-3}
             &              & Finger und Zehen sind gefühllos gegen Kälte und Schmerz\\ \cline{3-3}
             & 35 $^\circ$C & Muskeln werden steif und Bewegungen unkoordiniert\\ \hline
             &              & geistige Verwirrtheit, Gedächtnisstörung\\ \cline{3-3}
             & 34 $^\circ$C & Unvermögen, die Situation einzuschätzen\\ \cline{2-3}
             &              & verwirrtes Reden, verminderte Ansprechbarkeit\\ \cline{3-3}
  2. Stadium & 33 $^\circ$C & starkes Verlangen nach Schlaf\\ \cline{3-3}
             &              & Zittern kann aufhören\\ \cline{3-3}
             &              & Verlust des Kontakts zur Realität\\ \hline
             &              & Halluzination, Benommenheit\\ \cline{3-3}
  3. Stadium & 30 $^\circ$C & Puls und Atmung verlangsamt und schwach\\ \cline{3-3}
             &              & Bewusstlosigkeit\\ \hline
             &              & starkes Verlangen nach Schlaf\\ \cline{3-3}
  4. Stadium & 28 $^\circ$C & Herzrhythmusstörungen\\ \cline{3-3}
             &              & Herzstillstand, Lähmung des Atemzentrums, Tod\\ \hline
  \end{tabular}
\end{center}

\newpage
\section{Mein Text}

% Hier fügen Sie Ihren Text ein.

% [X] Text soll sich in zwei Spalten aufteilen.
% [X] Jede Spalte hat eine Breite von 5cm.
% [X] Jeweils mindestens ein Wort ist um einen Geviertstrich nach oben
%     bzw. unten zu verschieben.
% [X] Genau eine Zeile des Textes soll (unabhängig vom tatsächlichen Inhalt)
%     eine Höhe von 2,5cm belegen. => Interpretation: Nicht in jeder Spalte.
% [X] Die Spalten werden nicht mittels zweispaltigem Satz sondern Boxen erstellt.
\begin{center}
  \textbf{Unterkühlung}\\
  \vspace{5mm}
  \parbox{5cm}{
    Unterkühlung bedeutet das \textbf{Absinken der Kerntemperatur} des Körpers
    \raisebox{-1em}{unter} 37$^\circ$C. Diese Kerntemperatur, um die sich
    die ganze \raisebox{1em}{körperinterne} Energieversorgung dreht, ist die
    Temperatur der lebenswichtigen inneren Organe \raisebox{0mm}[1.25cm][1.25cm]{in}
    Rumpf und Kopf. Sie liegt bei uns Menschen
  }
  \hspace{1em}
  \parbox[c]{5cm}{
    bekanntlich um 37$^\circ$C. Fällt sie \raisebox{1em}{nur} um 1-2 Grad, so
    sind die Koordination der Bewegungen, das  Denkvermögen und das
    \raisebox{-1em}{willkürliche} Handeln bereits erheblich eingeschränkt. Und
    sinkt sie um mehr als 3-4 Grad, so kann dies zur Bewusstlosigkeit führen und
    den Tod zur Folge haben.
  }
\end{center}


\end{document}
