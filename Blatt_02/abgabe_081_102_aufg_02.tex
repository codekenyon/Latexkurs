% Rieke, Ken    4234782
% Thill, Gregor 4260617

\documentclass{article}

\usepackage{ngerman}
\usepackage[utf8]{inputenc}
\usepackage[T1]{fontenc}
\usepackage{parskip}

\newtheorem{beh}{Behauptung}[subsection]
\newtheorem{sch}[beh]{Schluss}

% TODO:
% [X] Titel mit Überschrift und Autor
% [X] Abstract
% [X] Einleitung (!= Abstract!)
% [X] Untergliederung mit mind. zwei Abschnitten
% [X] mind. zwei verschiedene eigens definierte Theorem-Umgebungen
%     mit gemeinsamer Nummerierung (z.B. "These" und "Beispiel" bei
%     bei einem politischen Thema
% [X] Aufzählungen unterschiedlicher Art, auch verschachtelt
% [X] einfaches Literaturverzeichnis und entsprechende Zitate im Text
% [X] Querverweise in allen möglichen Variationen

\begin{document}
\title{Die Kunst der leeren Rede}
\author{
  Ken Rieke\\
  Gregor Thill
}
\date{04.11.2014}
\maketitle
\begin{abstract}\label{abstract}
  Dieser Artikel betrachtet das Phänomen des kompetenten Auftretens bei
  vollkommener Ahnungslosigkeit und versucht, verschiedene Grundvoraussetzungen
  zu ermitteln und erläutern.
\end{abstract}

\section{Einleitung}\label{einleitung}
  Es wird wohl kaum eine Person geben, die noch nicht in eine Situation
  geraten ist, in der sie sich binnen kürzester Zeit - teilweise lediglich
  Sekunden - eine kompetent wirkende Aussage ausdenken musste, ohne jedoch die
  dazu notwendigen Fachkenntnisse aufzuweisen.

  In den nachfolgenden Abschnitten wird dieses Phänomen mit dem klangvollen
  Titel "`Kompetentes Auftreten bei vollkommener Ahnungslosigkeit"' detailliert
  betrachtet. Im Vordergrund steht dabei die Betrachtung der unten stehenden
  Punkte, deren Zuordnungen gleichzeitig die inhaltliche Struktur darstellen.

  \begin{itemize}
    \item Der Auftritt (Abschnitt \ref{auftritt})
    \begin{enumerate}
      \item Wie trete ich in einer solchen Situation auf?
      \item Vermeidung solcher Situationen
    \end{enumerate}
    \item Das Kaschieren (Abschnitt \ref{kaschieren})
    \begin{enumerate}
      \item Fachkenntnis vortäuschen
      \item Wenn im Trüben fischen nichts mehr nützt: Ablenkung
    \end{enumerate}
  \end{itemize}

\section{Der Auftritt}\label{auftritt}
  Wie bereits in der Einleitung auf Seite \pageref{einleitung} angegeben,
  befasst sich dieser Abschnitt mit der Beantwortung der Frage, wie jemand
  in der Situation auftreten kann, in der er fachlich überfordert ist.

  Anschließend wird noch kurz auf die Möglichkeit eingegangen, eine solche
  Situation schon vor ihrer Entstehung zu vermeiden.

  \subsection{Wie trete ich auf?}\label{auftritt_wie}
    \begin{beh}\label{ZeitSchinden}
      Eine Pause vor der Beantwortung einer Frage zeugt von Bedacht.
    \end{beh}
    Im Allgemeinen hat eine Person grundsätzlich zwei Möglichkeiten, auf eine
    Fachfrage zu reagieren, sobald Überreaktionen wie Schockstarre oder
    sofortige Abweisung ausgeschlossen werden:
    \begin{itemize}
      \item Eine schnelle Antwort, bei der sie sich um Kopf und Kragen redet.
      \item Eine verzögerte Antwort, die nach einiger Bedenkzeit geäußert wird.
    \end{itemize}
    Während die erste Möglichkeit aus offensichtlichen Gründen vermieden
    werden sollte, bietet die zweite Möglichkeit einen optimalen Ausweg aus
    der Situation. Die betroffene Person hat nicht nur Gelegenheit, den Schock
    zu verarbeiten, sondern kann sich noch eine an den Haaren herbeigezogene
    Geschichte ausdenken, die auf die Zuhörer jedoch so wirken kann, als würde
    aus einem sehr großen Wissensfundus geschöpft.
    \begin{sch}
      Das Fehlen von Hektik vermittelt den Eindruck von Kompetenz.
    \end{sch}

  \subsection{Vermeidung}\label{auftritt_vermeidung}
    Damit sich die Frage nach dem richtigen Auftritt (siehe \ref{auftritt_wie})
    gar nicht erst stellt, hat der bekannte Autor Un B. Kant eine Reihe von
    Tipps und Tricks zusammengetragen, die er mit der Anmerkung
    \begin{quote}
      Das ist eine sehr gute Frage, die ich Ihnen gern mit der dem Thema
      innewohnenden Ernsthaftigkeit beantworten möchte.
    \end{quote}
    in seinem Hauptwerk einleitet \cite[Seite XII]{ubk-ekg}.

    Aufgrund der urheberrechtlichen Sachlage können diese Tipps hier jedoch
    nicht aufgeführt werden.

\section{Das Kaschieren}\label{kaschieren}
  Nachdem das Auftreten in dieser teilweise äußerst unangenehm wirkenden
  Situation bereits in Abschnitt \ref{auftritt} behandelt wurde, stehen hier
  nun Möglichkeiten im Vordergrund, die Situation zu eigenen Gunsten zu retten.

  \subsection{Täuschung}\label{kaschieren_taeuschung}
    \begin{beh}
      "`Gut getäuscht ist fast gewonnen!"' \cite{bu-iwl}
    \end{beh}
    Wer die Täuschung meistert, ist nicht nur in der Lage, jederzeit kompetent
    aufzutreten, sondern vor allem ein guter Zuhörer, da sich nur dann
    authentisch wirkende Luftschlösser bauen lassen, sobald bekannt ist, wo
    die Zuhörer ihre Schwächen haben. Sind die jeweiligen Schwächen dem Redner
    bekannt, kann der in dieses, allen unbekannte, Terrain abwandern und dort
    die fantasievollsten Gebilde schaffen.

    Die Anwendung dieser Gesprächstaktik sollte aufgrund ihrer spezifischen
    Risiken auf einmalige Veranstaltungen beschränkt bleiben.
    \begin{sch}
      Eine Täuschung kann gelingen, ist jedoch aufwändig.
    \end{sch}

  \subsection{Ablenkung}\label{kaschieren_ablenkung}
    Die Ablenkung ist eine nahe Verwandte der Täuschung, bietet dem Redner
    jedoch die Möglichkeit, auf ein Gebiet abzudriften, auf dem tatsächlich
    eine gewisse Kompetenz vorhanden ist. Dies führt nicht nur zu einem
    wesentlich verbesserten Selbstwertgefühl, da man nicht mit dem Lügengebilde
    der Täuschung leben muss, sondern bietet auch eine Chance,
    abwechslungsweise nicht über die Arbeit zu reden.
    \begin{sch}
      Besser mit Hingabe über das eigene Hobby lamentieren als verunsichert
      ein fachliches Fettnäpfchen nach dem anderen zu treffen.
    \end{sch}

    \section{Fazit}\label{fazit}
  Mindestens ein großer Teil der Menschen, die in eine in Abschnitt
  \ref{einleitung} beschriebene Situation geraten, wird durch die hier
  aneinandergereihten Sätze kaum einen Erkenntnisgewinn abringen können. Dafür
  können diese aber nichts.

\begin{thebibliography}{[11]}
  \bibitem{ubk-ekg} {\slshape Eine kurze Geschichte\ldots}, Un B. Kant
  \bibitem{bu-iwl} {\slshape Die Irren und Wirren des Laberns}, Ben Utzer
\end{thebibliography}
\end{document}
