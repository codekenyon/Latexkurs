% Rieke, Ken    4234782
% Thill, Gregor 4260617

\documentclass{article}

\usepackage{ngerman}
\usepackage[utf8]{inputenc}
\usepackage[T1]{fontenc}

\begin{document}
\begin{center}
  {\large\scshape\bfseries Pizza Margherita mit grünem Salat}
\end{center}
\bigskip
{\bfseries\itshape Für 6 Personen}\\
\\
\small Für die Pizza\normalsize\newline
\\
\smallskip
20 g Hefewürfel\\
\smallskip
1 TL Zucker oder Honig\\
\smallskip
500 g Weizenvollkornmehl\footnote{Type 1050}\\
\smallskip
Salz\\
\smallskip
5 EL Olivenöl\\
\smallskip
700 g Tomaten\\
\smallskip
1 große Zwiebel\\
\smallskip
250 g Mozzarella\\
\smallskip
Pfeffer aus der Mühle\\
\smallskip
100 g italienische Kräutermischung\\
\smallskip
1 Knoblauchzehe, fein gehackt\\
\bigskip
\\
%Beschreibung der Zubereitung
{\Huge\textbf{1}}
  Hefe zerbröckeln und \slshape Zucker \upshape oder \slshape Honig \upshape
  dazugeben.\\
  Aufgelöste Hefe über das \slshape Mehl \upshape geben.\\
  \bigskip
  \\
{\huge\textbf{2}}
  Etwas {\sffamily Wasser}, \slshape Salz \upshape und \slshape 1 Esslöffel
  Ölivenöl \upshape dazugeben\\
  und alles zu einem elastischen Teig kneten.
  \begin{quote}
    {\it \textbf{(Achtung}, die Hefe darf nicht direkt mit dem Öl und dem Salz
    in Berührung kommen - sie verliert sonst ihre Triebkraft. Also erst die
    Hefe mit etwas Mehl vermischen.)}\\
  \end{quote}
  Den Hefeteig zugedeckt ca. 30 Minuten gehen lassen.\\
  \bigskip
  \\
{\LARGE\textbf{3}}
  In der Zwischenzeit \slshape Tomaten \upshape in Scheiben und \slshape
  Zwiebeln \upshape in feine Ringe schneiden, \slshape Mozzarella \upshape
  zerzupfen.\\
  \bigskip
  \\
{\Large\textbf{4}}
  Teig durchkneten und zu einem dünnen Boden ausrollen. Mit \slshape Tomaten,
  Zwiebeln \upshape und \slshape Mozzarella \upshape belegen, mit \slshape
  Salz,\upshape Pfeffer \upshape und der \slshape Kräutermischung \upshape
  würzen. \slshape Knoblauch \upshape mit dem restlichen \slshape Ölivenöl
  \upshape vermischen und über die Pizza träufeln.

  Bei 200 C ca. 25 Min bis 30 Minuten backen.\newpage
\small Für den Salat\normalsize\newline
\\
\medskip
1 Zwiebel\\
\medskip
1 EL Speiseleinöl\\
\medskip
1 EL Olivenöl\\
\medskip
200 g Hüttenkäse\\
\medskip
1,5 Essig\\
\medskip
1 TL Hefeflocken\\
\medskip
1 TL mittelscharfer Senf\\
\medskip
1 TL Meerrettich\\
\medskip
50 g Walnüsse, gehackt\\
\medskip
Salz, Pfeffer aus der Mühle\\
\medskip
Italienische Kräutermischung\\
1 kg Kopfsalat\\
\bigskip
\\
{\large\textbf{5}}
  Für den Salat die \slshape Zwiebeln \upshape in kleine Würfel schneiden\\
  und für ca. 5 Minuten ruhen lassen.\\
\bigskip
\\
{\normalsize\textbf{6}}
  \slshape Lein- \upshape und \slshape Olivenöl, Hüttenkäse, Essig,
  Hefeflocken, Senf, Meerrettich, \upshape und \slshape Walnüsse \upshape
  sowie \slshape Salz, Pfeffer \upshape und \slshape Kräuter \upshape zu einer
  pikanten Salatsauce vermischen.\\
  \bigskip
  \\
{\tiny\textbf{7}}
  Den zerzupften \slshape Kopfsalat \upshape erst kurz vor dem Servieren in die
  Salatsauce geben.\\
  \bigskip
\pagebreak
\\
{\Huge\ttfamily\textbf{Tipp}}
  \sffamily Als Nachtisch gibt es pro Person 1 Apfel.\rmfamily\\
  \smallskip
  \\
{\bf Zubereitung: ca. 50 Minuten (+Gehzeit des Teigs)}\\
{\textbf{Nährwert pro Person: }\textit{633 kcal, Kohlenhydratanteil 45\%,
  Eiweiß 18\%, Fett 37\%}
\end{document}
