% Rieke, Ken    4234782
% Thill, Gregor 4260617

\documentclass[12pt,a4paper]{scrartcl}
\usepackage[utf8]{inputenc}
\usepackage[T1]{fontenc}
\usepackage[ngerman]{babel}
\usepackage{amsmath,amssymb}
\setlength{\parindent}{0mm}

%%%%%%%%%%%%%%%%%%
% Eigene Befehle %
%%%%%%%%%%%%%%%%%%
\newcounter{AufgabeNr}
\newcounter{TeilaufgNr}[AufgabeNr]
\newcounter{Punkte}[AufgabeNr]

\setcounter{AufgabeNr}{0}
\setcounter{TeilaufgNr}{0}
\setcounter{Punkte}{0}

\newenvironment{Aufgabe}[1]
{
  \stepcounter{AufgabeNr}
  \par
  \textbf{Aufgabe \thesection.\theAufgabeNr\ (#1)}
  \par
}
{
  \par
  \hfill Gesamtpunktzahl bei Aufgabe \thesection.\theAufgabeNr: \thePunkte\ P.    
}

\newsavebox{\Ende}
\newenvironment{Teilaufg}[2][!]
{\sbox{\Ende}{\textbf{#1}}
  \stepcounter{TeilaufgNr}
  \addtocounter{Punkte}{#2}
  \par
  \textit{Teilaufgabe \thesection.\theAufgabeNr.\roman{TeilaufgNr} (#2 P.):}
}
{
  \hfill\usebox{\Ende}\par
}

%%%%%%%%%%%%%%%%%%%%%%%%
% Ende eigener Befehle %
%%%%%%%%%%%%%%%%%%%%%%%%

\begin{document}

\section{Eingabe}
\begin{Aufgabe}{Geräte}
  Umfangreicher langer Text, damit die Zeile auch voll wird und ein Zeilenumbruch entsteht, dies sollte nun aber reichen.
  \begin{Teilaufg}[1]{1}
Umfangreicher langer Text, damit die Zeile auch voll wird und ein Zeilenumbruch entsteht, dies sollte nun aber reichen.
  \end{Teilaufg}
  \begin{Teilaufg}{1}
Umfangreicher langer Text, damit die Zeile auch voll wird und ein Zeilenumbruch entsteht, dies sollte nun aber reichen.
  \end{Teilaufg}
  \begin{Teilaufg}[?]{2}
Umfangreicher langer Text, damit die Zeile auch voll wird und ein Zeilenumbruch entsteht, dies sollte nun aber reichen.
  \end{Teilaufg}
\end{Aufgabe}

\begin{Aufgabe}{Datenformate}
  Aufgabentext
  \begin{Teilaufg}{3}
    Teilaufgabentext
  \end{Teilaufg}
\end{Aufgabe}

\section{Verarbeitung}

\section{Ausgabe}

\setcounter{section}{41}
\section{Schnittstellen}

\section{Verbundstellen}

\end{document}
