% Rieke, Ken    4234782
% Thill, Gregor 4260617

\documentclass[12pt,a4paper]{scrartcl}
\usepackage[utf8]{inputenc}
\usepackage[T1]{fontenc}
\usepackage[ngerman]{babel}
\usepackage{amsmath,amssymb}
\setlength{\parindent}{0mm}

%%%%%%%%%%%%%%%%%%
% Eigene Befehle %
%%%%%%%%%%%%%%%%%%
\newcounter{AufgabeNr}
\newcounter{TeilaufgNr}[AufgabeNr]
\newcounter{Punkte}[AufgabeNr]

\setcounter{AufgabeNr}{0}
\setcounter{TeilaufgNr}{0}
\setcounter{Punkte}{0}

\newenvironment{Aufgabe}[1]
{
  \stepcounter{AufgabeNr}
  \par
  \textbf{Aufgabe \thesection.\theAufgabeNr\ (#1)}
  \par
}
{
  \par
  \hfill Gesamtpunktzahl bei Aufgabe \thesection.\theAufgabeNr: \thePunkte\ P.
}

\newsavebox{\Ende}
\newenvironment{Teilaufg}[2][!]
{\sbox{\Ende}{\textbf{#1}}
  \stepcounter{TeilaufgNr}
  \addtocounter{Punkte}{#2}
  \par
  \textit{Teilaufgabe \thesection.\theAufgabeNr.\roman{TeilaufgNr} (#2 P.):}
}
{
  \hfill \usebox{\Ende}\par
}

%%%%%%%%%%%%%%%%%%%%%%%%
% Ende eigener Befehle %
%%%%%%%%%%%%%%%%%%%%%%%%

\begin{document}

\section{Eingabe}
Ein EDV-System bedient sich zahlreicher Peripheriegeräte für die Kommunikation
mit anderen System und dem Anwender. Dieser Testabschnitt befasst sich einerseits
mit Geräten der Eingabe, anderseits auch mit den dafür verwendeten Formaten.
\begin{Aufgabe}{Geräte}
  Die folgenden Aufgaben fragen Ihre Kenntnisse im Bereich der Eingabegeräte ab.
  \begin{Teilaufg}{5}
    Nennen und beschreiben Sie mindestens fünf unterschiedliche Eingabegeräte und
    ordnen Sie diese, wo möglich, ihrem optimalen Einsatzgebiet zu.
  \end{Teilaufg}
  \begin{Teilaufg}[!Nicht schmulen!]{15}
    Verfassen Sie einen kurzen Bericht über eines der zuvor genannten Eingabegeräte,
    in dem Sie Vorteile, Nachteile und die Einordnung in das technische Umfeld
    erläutern.
  \end{Teilaufg}
  \begin{Teilaufg}[?]{2}
    Schreiben Sie Ihr Lieblingswort auf und ersetzen Sie darin vorkommende i-Punkte
    durch ein Symbol Ihrer Wahl.
  \end{Teilaufg}
\end{Aufgabe}

\begin{Aufgabe}{Datenformate}
  Die Umwandlung von durch Menschen handhabbare Daten in eine für EDV-Systeme
  verarbeitbare Form ist Hauptgegenstand dieser Aufgabenstellung.
  \begin{Teilaufg}[Aufgepasst]{20}
    Erzeugen Sie aus den Angaben auf dem Beiblatt einen Eingabetext im vereinfachten
    AsSc-Format. \emph{Bedenken Sie die strikten Regeln für Umlaute und Sonderzeichen}.
  \end{Teilaufg}
  \begin{Teilaufg}{11}
    Tauschen Sie Ihr bisheriges Aufgabenblatt mit Ihrem linken Nachbarn und überprüfen
    Sie die bisher verfassten Lösungen auf Rechtschreibfehler.
  \end{Teilaufg}
\end{Aufgabe}

\section{Verarbeitung}
Die Verarbeitung der Daten erfolgt abhängig vom technischen Design auf mannigfaltige
Art und Weise.
\begin{Aufgabe}{Algorithmen}
  \begin{Teilaufg}{5}
    Beschreiben Sie, was ein Algorithmus ist.
  \end{Teilaufg}
  \begin{Teilaufg}[Wohlfühlfrage]{5}

    Nennen Sie drei unterschiedliche Algorithmen und beobachten danach ausgiebig
    \textbf{(mindestens 5 Minuten lang)} die Schmetterlinge auf der Wiese neben
    dem Gebäude.
  \end{Teilaufg}
  \begin{Teilaufg}[Wahlaufgabe]{13}
    Sollten Sie bisher nicht von Ihrem rechten Nachbarn zum Tausch Ihrer Prüfung
    aufgefordert worden sein, zeichnen Sie an dieser Stelle das Haus des Nikolaus
    -- ohne abzusetzen!

  \end{Teilaufg}
\end{Aufgabe}
\begin{Aufgabe}{Architekturen}
  Obwohl natürlich ein EDV-System als Ganzes auch eine bestimmte Architektur
  repräsentiert, gibt es auch für den Bereich der Datenverarbeitung eine
  ganze Reihe mehr oder weniger stark verbreiteter Architekturen.
  \begin{Teilaufg}[Hä?!]{15}

    Beschreiben Sie fünf unterschiedliche Architekturen der Datenverarbeitung
    und unterstreichen Sie die zwei, welche Ihrer Meinung nach am besten für
    den Einsatz in verteilten Systemen geeignet sind.

  \end{Teilaufg}
  \begin{Teilaufg}[99]{11}
    Analysieren Sie eine der beiden favorisierten Architekturen in Bezug auf
    ihre Wirtschaftlichkeit.
  \end{Teilaufg}
  Der Sachbearbeiter für diesen Aufgabenkomplex ist derzeit im Ausland, weshalb diese
  Aufgabe nicht korrigiert werden kann und deshalb mit 0 bewertet wird. Beschwerden
  dagegen führen zu weiteren Abzügen.
\end{Aufgabe}
\begin{Aufgabe}{Mythen}
\end{Aufgabe}

\section{Ausgabe}
\begin{Aufgabe}{Ausgabeformate}
  \begin{Teilaufg}[iddqd]{19}
  \end{Teilaufg}
\end{Aufgabe}

\end{document}
