% Rieke, Ken    4234782
% Thill, Gregor 4260617

\documentclass[12pt,a4paper]{scrartcl}
\usepackage[utf8]{inputenc}
\usepackage[T1]{fontenc}
\usepackage[ngerman]{babel}
\usepackage{amsmath,amssymb}

% Farben und so
\usepackage{color}

\begin{document}
\setlength{\unitlength}{3ex}
\begin{picture}(26,23)
  % Waagerechte Pfeile
  \multiput(0,7)(0,6){3}{
    \multiput(0,0)(6,0){4}{
      \put(0,0){\vector(1,0){2}}
    }
  }
  % Senkrechte Pfeile (schwarz)
  \multiput(4,23)(0,-6){3}{
    \multiput(0,0)(6,0){4}{
      \put(0,0){\vector(0,-1){2}}
    }
  }
  % Senkrechte Pfeile (unterste Zeile, rot)
  \multiput(4,5)(6,0){4}{
    \put(0,0){{\color{red}\vector(0,-1){2}}}
  }
  % Kreise mit e
  \multiput(4,19)(0,-6){3}{
    \put(0,0){\circle{3}\makebox(0,0){e}}
  }
  % Kreise mit m
  \multiput(10,19)(0,-6){3}{
    \put(0,0){\circle{3}\makebox(0,0){m}}
  }
  % Kreise mit (grünem) c
  \multiput(16,19)(0,-6){3}{
    \put(0,0){\circle{3}\makebox(0,0){{\color{green}c}}}
  }
  % Kreise mit c
  \multiput(22,19)(0,-6){3}{
    \put(0,0){\circle{3}\makebox(0,0){c}}
  }
  % Blaue Pfeile
  \multiput(6,12)(0,6){2}{
    \put(0,0){{\color{blue}\vector(1,-1){3}}}
    \put(0,0){{\color{blue}\qbezier(0,0)(4,-4)(8,0)}}
    \put(8,0){{\color{blue}\vector(1,1){0}}} % Pfeilspitze für Kurve
  }
  % "Klammern" am rechten Rand
  \multiput(24,10)(0,6){2}{
    \put(0,0){{\thicklines\oval(4,6)[r]}}
  }
  % Rote Kurve am rechten Rand
  \put(23,8){{\color{red}\qbezier(0,0)(2,6)(0,10)}}
  % Oval mit Schriftzug e = mc^2?
  \put(13,1){{\thicklines\color{red}\oval(20,2)\makebox(0,0){$e = mc^2\ ?$}}}
\end{picture}

\end{document}
