% Rieke, Ken    4234782
% Thill, Gregor 4260617

\documentclass[12pt,a4paper]{scrartcl}
\usepackage[utf8]{inputenc}
\usepackage[T1]{fontenc}
\usepackage[ngerman]{babel}
\usepackage{amsmath,amssymb}

% Farben und so
\usepackage{color}

% Später wieder entfernen!
\usepackage{graphpap}
\definecolor{Gitternetz}{gray}{0.8}
\newenvironment{grafik}[3][2]{
  \setlength{\unitlength}{3ex}
  \begin{picture}(#2,#3)
    \color{Gitternetz}
    \graphpaper[#1](0,0)(#2,#3)
    \color{black}
}{
  \end{picture}
}
% Schluss mit Entfernen!


\begin{document}
%\setlength{\unitlength}{3ex}                                    % Aktivieren nicht vergessen!
%\begin{picture}(26,23)                                          % Aktivieren nicht vergessen!
\begin{grafik}{26}{23}                                           % Löschen nicht vergessen
  % Waagerechte Pfeile
  \multiput(0,7)(0,6){3}{
    \multiput(0,0)(6,0){4}{
      \put(0,0){\vector(1,0){2}}
    }
  }
  % Senkrechte Pfeile (schwarz)
  \multiput(4,23)(0,-6){3}{
    \multiput(0,0)(6,0){4}{
      \put(0,0){\vector(0,-1){2}}
    }
  }
  % Senkrechte Pfeile (unterste Zeile, rot)
  \multiput(4,5)(6,0){4}{
    \put(0,0){{\color{red}\vector(0,-1){2}}}
  }
  % Kreise mit e
  \multiput(4,19)(0,-6){3}{
    \put(0,0){\circle{3}\makebox(0,0){e}}
  }
  % Kreise mit m
  \multiput(10,19)(0,-6){3}{
    \put(0,0){\circle{3}\makebox(0,0){m}}
  }
  % Kreise mit (grünem) c
  \multiput(16,19)(0,-6){3}{
    \put(0,0){\circle{3}\makebox(0,0){{\color{green}c}}}
  }
  % Kreise mit c
  \multiput(22,19)(0,-6){3}{
    \put(0,0){\circle{3}\makebox(0,0){c}}
  }
  % Blaue Pfeile
  \multiput(6,12)(0,6){2}{
    \put(0,0){{\color{blue}\vector(1,-1){3}}}
    \put(0,0){{\color{blue}\qbezier(0,0)(4,-4)(8,0)}}
    \put(8,0){{\color{blue}\vector(1,1){0}}} % Workaround für Pfeilspitze, bis bessere Lösung in Sicht.
  }
  % "Klammern" am rechten Rand
  \multiput(24,10)(0,6){2}{
    \put(0,0){{\thicklines\oval(4,6)[r]}}
  }
  % Rote Kurve am rechten Rand
  \put(23,8){{\color{red}\qbezier(0,0)(2,6)(0,10)}}
  % Oval mit Schriftzug e = mc^2?
  \put(13,1){{\thicklines\color{red}\oval(20,2)\makebox(0,0){$e = mc^2\ ?$}}}
\end{grafik}                                                     % Löschen nicht vergessen!
%\end{picture}                                                   % Aktivieren nicht vergessen!

\end{document}
