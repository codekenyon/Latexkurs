% Rieke, Ken    4234782
% Thill, Gregor 4260617

\documentclass[ngerman]{beamer}
% Fix für Warnungen Nr. 3 und 4 dargestellt in:
% http://texblog.net/latex-archive/latex-general/beamer-warnings/
\let\Tiny=\tiny
\usepackage[utf8]{inputenc}
\usepackage[T1]{fontenc}
\usepackage[ngerman]{babel}

\deftranslation[to=ngerman]{Theorem}{Satz}
\deftranslation[to=ngerman]{Proof}{Beweis}
\deftranslation[to=ngerman]{Corollary}{Folgerung}
\deftranslation[to=ngerman]{Example}{Beispiel}

\author{Ken Rieke \and Gregor Thill}
\institute{Technische Universität Carolo-Wilhelmina zu Braunschweig}
\date{\today}

\begin{document}
\frame{\titlepage}

\begin{frame}[label=inhalt]
  \frametitle{Inhalt}
  \tableofcontents[pausesections]
\end{frame}

\section{Einfache Foliengestaltung}
\subsection{Alle Punkte zugleich}
\begin{frame}[label=paritaet]
  \frametitle{Parität}
  \framesubtitle{im mathematischen Sinne}
  \begin{definition}
    Eine ganze Zahl heißt \alert{gerade}, wenn sie ohne Rest durch zwei teilbar
    ist; andernfalls heißt sie \alert{ungerade}.\cite{wikiped-1}
  \end{definition}
  \begin{example}
    \begin{itemize}
      \item $42 / 2 = 21$ R $0 \Rightarrow 42$ ist gerade
      \item $21 / 2 = 10$ R $1 \Rightarrow 21$ ist ungerade
      \item $10 / 2 = 5$ R $0 \Rightarrow 10$ ist gerade
      \item $5 / 2 = 2$ R $1 \Rightarrow 5$ ist ungerade
    \end{itemize}
  \end{example}
\end{frame}

\subsection{Ein Punkt nach dem Anderen}
\begin{frame}[label=reihe]
  \frametitle{Ein Punkt nach dem Anderen}
  \framesubtitle{Steuerung des Fokus}
  \begin{theorem}
    Zuerst wird dieser Satz aufgedeckt.
  \end{theorem}\uncover<2->{
  \begin{proof}
    \begin{itemize}
      \item<2-> Beweisanfang
      \item<3-> Beweisfortführung
      \item<4-> Ablenkung auf Folie \ref{inhalt}
      \item<5-> Beweisende\qedhere
    \end{itemize}
  \end{proof}}\uncover<6->{
  \begin{corollary}
    Nebenprodukt des soeben geführten Beweises.
  \end{corollary}}
\end{frame}

\section{Komplexere Foliengestaltung}
\subsection{Großes Durcheinander}
\begin{frame}
  \frametitle{Großes Durcheinander}
  \framesubtitle{Fokusführung für Fortgeschrittene}
  \begin{block}{Wiederholung...}
    ...der Folien \ref{paritaet} bis \ref{reihe}.
  \end{block}\uncover<2->{
  \begin{itemize}
    \item<2,4,6,7> Das
    \item<1-3,7> wäre
    \item<4-6,7> doch
    \item<2-4,7> zu
    \item<1,3,5,7> einfach.
  \end{itemize}}
\end{frame}

\subsection{Only one}
\begin{frame}[t,plain,fragile]
  \frametitle{Only one}
  \framesubtitle{Der springende Punkt}
  Spielerei mit \verb:\only:

  \only<1>{Eins\\}
  \only<2-3>{Zwei bis Drei\\}
  \only<-4>{bis Vier}

  Und auch mit den Flags t, plain sowie fragile.
\end{frame}

\section{Quellen}
\begin{frame}
  \frametitle{Quellen}
  \begin{thebibliography}{9}
    \bibitem{wikiped-1}
      Verschiedene,
      \emph{Parität (Mathematik)}.\\
      https://de.wikipedia.org/wiki/Parität\_(Mathematik),
      abgerufen am 20. Januar 2015
  \end{thebibliography}
\end{frame}
\end{document}
