% Rieke, Ken    4234782
% Thill, Gregor 4260617

\documentclass[ngerman]{beamer}
\usepackage[utf8]{inputenc}
\usepackage[T1]{fontenc}
\usepackage[ngerman]{babel}

\deftranslation[to=ngerman]{Theorem}{Satz}
\deftranslation[to=ngerman]{Proof}{Beweis}
\deftranslation[to=ngerman]{Corollary}{Folgerung}
\deftranslation[to=ngerman]{Example}{Beispiel}

% TODO
% [ ] Ein Inhaltsverzeichnis, das schrittweise aufgedeckt wird
% [ ] Mindestens eine Folie, die nach und nach und nicht nur von oben nach unten aufgedeckt wird.
% [ ] Mindestens je einen Satz, eine Folgerung und einen Beweis (mit deutscher Bezeichnung)
% [ ] Mindestens einen Querverweis auf die Nummer einer anderen Folie.
% [ ] Layout- und Farbschema, das nicht der Vorlesungs-Präsentation entspricht

\begin{document}
\author{
  Ken Rieke \and Gregor Thill}
\institute{Technische Universität Carolo-Wilhelmina zu Braunschweig}
\date{\today}

\begin{frame}
  \titlepage
\end{frame}

\begin{frame}
  \frametitle{Inhalt}
  \tableofcontents[pausesections]
\end{frame}

\section{Einfache Foliengestaltung}
\subsection{All at once}
\begin{frame}
  \frametitle{All at once}
  \framesubtitle{lenkt Betrachter ab}
  \begin{theorem}
    Blabla
  \end{theorem}
  \begin{proof}
    Blabla
  \end{proof}
  \begin{lemma}
    Blabla
  \end{lemma}
\end{frame}

\subsection{Fading in order}
\begin{frame}
  \frametitle{Fading in order}
  \framesubtitle{Steuerung des Fokus}
  \begin{definition}
    Blabla
  \end{definition}
  \begin{example}
    Blabla
  \end{example}
\end{frame}

\section{Komplexere Foliengestaltung}
\subsection{Fading out of order}
\begin{frame}
  \frametitle{Fading out of order}
  \framesubtitle{Verbesserter Fokus}
  Dies ist ein wenig Text.
\end{frame}

\subsection{Only one}
\begin{frame}
  \frametitle{Only one}
  \framesubtitle{Der springende Punkt}
  Dies ist ein wenig Text.
\end{frame}

\end{document}
