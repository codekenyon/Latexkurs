% Rieke, Ken    4234782
% Thill, Gregor 4260617

\documentclass[12pt,a4paper]{scrartcl}
\usepackage[ngerman]{babel}
\usepackage[utf8]{inputenc}
\usepackage[T1]{fontenc}

\usepackage{amsmath}
\usepackage{amssymb}

\DeclareMathOperator{\grad}{grad}
\newcommand{\Sum}{\sum\limits_{i=1}^{n}}
\newcommand{\abMinus}{{a_{i}b_{i}}^{-1}}
\newcommand{\baMinus}{{b_{i}a_{i}}^{-1}}
\newcommand{\fivedots}{\cdot \cdots \cdot}

\begin{document}
\thispagestyle{empty}

Mathematische Formeln sind recht leicht umzusetzen. Für den Binomialkoeffizienten
gilt die Aussage
\[
  \binom{n}{k}
  = \frac{n}{1} \cdot \frac{n-1}{2} \cdot \ldots \cdot \frac{n-k+1}{k}
  = \frac{n \cdot (n-1) \fivedots (n-k+1)}{k!}
  = \prod_{i=1}^{k} \frac{n+1-i}{i}.
\]

\noindent
Es sei $f(x_{1}, \ldots, x_{n})$ für $x_{i} \in \mathbb{R}$ eine Funktion. Ein bestimmter
Eintrag im Gradienten dieser Funktion, d.h. die partielle Ableitung in $x_{i}$-Richtung,
sieht dann so aus:
\[
 (\grad f)_{i}
 = \frac{\partial f}{\partial x_{i}}.
\]

\noindent
Allgemein gilt $\alpha \grad f + \beta \grad g = \grad(\alpha f + \beta g)$. Integrale
können ebenso spannend sein, wie aus der Funktion
\[
 f(z)
 = \frac{1}{2 \pi \mathrm{i}}
   \int\limits_{M} \frac{f(\zeta)}{\zeta-z} \mathrm{d}\zeta
   \qquad \text{ für alle } z \in \mathfrak{K}(z_{0};r)
\]

\noindent
ersichtlich ist. Sicherlich kennen Sie auch $\sqrt{1-(\sin x)^2} = \cos x$. Zuletzt folgt noch
eine lange Formel, die u.a. dem Anwenden eigener Befehle dienen soll:
\[
  \Sum\abMinus + \Sum\baMinus
  = \Sum(\abMinus + \baMinus)
  \geq \Sum 2 =2n.
\]
\end{document}
